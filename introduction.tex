%% Plato FSM Translation Tech Memo
%% Introduction

\section{Introduction \label{sec:intro}}

As discussed in~\cite{2015_Beaumont_MEMOCODE}, concepts are a useful language
for specifying the behaviours of asynchronous circuits,
in the preferred form of the user. This can be as
low-level signal-level concepts, or higher-level gate or
protocol-level concepts. It also allows the definition
of their own concepts, which can be reused within
the same or any other specification that they wish, to
increase the speed of designing a system, and future
systems.

In~\cite{2016_concept_STG_translation}, we introduce an 
algorithm for automatically translating concepts to Signal Transition Graphs (STGs). 
This is aimed at solving problems that can arise with the monolothic approach
for desigining asynchronous circuits with STGs, such as a lack of reusability, and 
poor scalability. 

Since STGs are so commonly used for desinging asynchronous circuits,
there are several tools available, such as \noun{Petrify}~\cite{Cortadella}, \noun{Mpsat}~\cite{khomenko2004detecting}, \noun{Versify}~\cite{i1997formal},
\noun{Workcraft}~\cite{Sokolov-2016-book-Workcraft}\cite{Workcraft_website}, and others, which can automatically verify and synthesize
asynchronous circuits. Automated translation of concepts to STGs therefore allows concept specifications
to be usable with these tools, avoiding the need to develop verification and synthesis tools specifically for 
concepts, which is a timely process. 

Finite State Machines (FSMs) can be used for a wide range of applications, from small scale
to large scale, low-level to high-level system designs. Because of this, they are commonly taught
as a method of designing overall system operation. This means that many designers in industry
are aware of of FSMs, and are more likely to understand them than STGs, which are developed
and primarily used in academia. For this reason, it may be useful when specifying an asynchronous circuit 
using concepts, to be able to view the system as an FSM, and automatic translation will allow this to occur quickly.

However, while viewing the FSM may be useful for a user, STG tools are still the primary methods of 
verifiying and synthesizing asynchronous circuits. The concepts language does not provide any
concepts only for use when translating to either an FSM or an STG specifically, so any concept specification
which translates to one of these will also produce an equivalent model of the alternative formalism, with
no changes to the specification. Each translation algorithm handles the differences in the two formalisms,
producing the correct construcs for that formalism, based on the provided concepts. 

The open-source command-line tool \noun{Plato}~\cite{2017_plato_github}, is authoured in Haskell, and implements the domain-specific language of concepts, 
as well as the algorithms for translating this language to both STGs and FSMs. \noun{Plato} is included as a
back-end tool for \noun{Workcraft}~\cite{Sokolov-2016-book-Workcraft}\cite{Workcraft_website}, where it is integrated to provide
a graphical user interface for authoring and translating concepts. It also includes STG verification and synthesis tools, thus providing
a streamlined interface for designing asynchronous systems from concept specification, to implementation. 

Our contributions are as follows:
\begin{itemize}

\item We discuss Finite State Machines, their uses, their differences from STGs and how concepts can be used to produce FSMs in Section~\ref{sec:fsms}

\item We present the implemented algorithm for translating
concepts to FSMs in Section~\ref{sec:algorithm}.

\item We present a brief example of the design flow in Section~\ref{sec:design-flow}

\end{itemize}