%% Plato FSM Translation Tech Memo
%% FSM background
\newpage
\section {Finite State Machines \label {sec:fsms}}

Finite State Machines can be used to model behaviour of all levels of system~\cite{Taraate2016}. 
The formalism contains states, which are the vertices, and transitions between these states, which 
are the arcs. Applied to each transition is the event for that transition to occur, causing a change 
of state. Depending on the level of system that is being modelled, this event can be a condition being met,
such as the temperature being over a certain value, or it could be a signal transition. 

Regardless of system level, an FSM model always shows some key parts of the system. Each state a system is 
in usually has an associated output or action, which occurs upon entry to this state and continues while the system remains
in this state, and when an event occurs, the state will change depending on what state the system is in, and which event occured. 

For example, Figure~\ref{fig:stopwatch} contains the example FSM which models the 
high-level behaviour of a stopwatch.

\begin{figure}[h]
\begin{centering}
\includegraphics[scale=0.5]{images/stopwatch-fsm}
\par\end{centering}
\protect\caption{\label{fig:stopwatch} Stopwatch FSM}
\end{figure}

\noindent There are four states, which are:

\begin{itemize}

\item Zero - The initial state, where the timer and display are reset to 0. 

\item Stop - In this state, the timer and display are paused.

\item Run - While in this state, the timer is counting, and the display shows this.

\item Lap - In this state, the timer continues counting but the display shows the time when lap was entered.

\end{itemize}

\noindent The transitions between these are based on three buttons:

\begin{itemize} 

\item Reset button - When it has been stopped (in the stop state), moves the system to the zero state to reset everything to 0.

\item Start/Stop button - Used to move the stopwatch from the stop state to the run state, and back. This starts and pauses the timer. 

\item Lap button - This moves the system from the run state, to the lap state, to hold a time on the display for recording. 
                              Also moves it back to the run state, updating the display to the actual time
                              
\end{itemize}

\noindent Also, not that when the zero state is reached, there is an $\varepsilon$ symbol. This symbol represents that
a transition occurs without any requirements. For this example, it is important, as when the system initialises or is reset, we 
want the display and counter to be set to 0, and following this, be ready to start counting. This therefore resets the system, and
moves into the stop state, ready for the timer to being counting from a start/stop button press. 

This is a high-level model as it describes the possible states of the system, but not the low-level implementation details. 
For example, we know what buttons to press to move between states, but there is nothing to say what constitues a button ``press'', 
and, while we know how the display should react depending on the state, we have not provided any details on how we control the display. 

The stopwatch behaviour however does identify the key parts. Each state has an associated action, which we have described. Each transition
has an associated event, these being either button presses, or nothing. We also know that for each state, we can transition to a new state if a
specific button is pressed, but if that button is not associated with a transition from the current state, the state will remain the same, for example, 
if we press the reset button while in the run state, we cannot changes state. 

This example can be used further by adding some implementation details, based on how the components, the buttons and display, work, yet 
the behaviour of the stop-watch will remain the same. 

FSMs can be used to model circuits at the signal-level, as with STGs. In this case, the actions associated with the states can be simply the
arrangement of the signals in the system, the effect of this arrangment causing changes in the environment of the circuit itself. The events 
associated with the transitions between states can be signal transitions. For example, Figure~\ref{fig:handshake-fsm} contains a handshake FSM.

\begin{figure}[h]
\begin{centering}
\includegraphics[scale=0.5]{images/handshake-fsm}
\par\end{centering}
\protect\caption{\label{fig:handshake-fsm} Handshake FSM}
\end{figure}

In this example there is one input signal, $a$, one output signal, $b$, and four states: 

\begin{itemize}

\item 00 - The initial state, where both input and output have transitioned low.

\item 10 - The input has gone from 0 to 1.

\item 11 - The output transitions from 0 to 1.

\item 01 - The input transitions low, from 1 to 0.

\end{itemize}

\noindent The transitions between feature the events of these signals transitioning high and low. 

In this case, we have included the implementation details, specifically in what signal transitions cause a changes of state, 
therefore this can be considered a low-level FSM model. However, this doesn't change the fact that the key behaviours are identified. 

Since FSMs can be used generally at all different levels of behavioural modelling, the requirements for what constitutes and event for
a state transition, are not stated, therefore, as with these examples they can be a button press or a signal transition. There exists 
a form of FSM specifically designed for modelling circuit behaviour at signa-level, such as in Figure~\ref{fig:handshake-fsm} named 
\emph{Finite State Transducers} (FSTs). 

FSTs aim to be used for specifications of circuits, and can be used for specifying asynchronous circuits. They allow signals
to be set as specific types, inputs, outputs and internals for example, but do not differ in how they show behaviours. For this reason, 
they can be converted to and from other modelling methods aimed at showing signal-level behaviours such as STGs, and this applies
to \noun{Plato}, where a concept specification must be able to produce either an FSM or an STG without any changes to the specification. 

When using \noun{Workcraft}, an STG can be translated from a concept specification using \noun{Plato} from the STG plugin, and while \noun{Workcraft} does
feature both an FSM and FST plugin, for the interoperability of STGs and FSTs, a concept specification can be translated only to an FST. For this paper however, 
we will continue to FSTs translated from concepts as FSMs. 

\subsection{FSMs and STGs \label{sub:FSMs-STGs}}